\documentclass{article}
\usepackage[utf8]{inputenc}
\usepackage{graphicx}

\title{Métodos Computacionales - Tarea 4}
\author{María Daniela Muñoz}
\date{Noviembre 2017}

\begin{document}

\maketitle

\section{Cuerda}

\begin{figure}[h]
\begin{center}
\includegraphics[height=0.7\textwidth]{cuerda.png}
\caption{Evolución de una cuerda que ha sido perturbada en el punto medio para t=0 y que tiene los extremos fijos.} \label{cuerda}
\end{center}
\end{figure}

\begin{figure}[h]
\begin{center}
\includegraphics[height=0.7\textwidth]{cuerdaPerturbada.png}
\caption{Evolución de una cuerda que ha sido perturbada desde el extremo derecho para t=0.} \label{cuerdaP}
\end{center}
\end{figure}



\clearpage
\section{Tambor}

\end{document}
